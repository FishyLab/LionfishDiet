\documentclass[]{article}
\usepackage{lmodern}
\usepackage{amssymb,amsmath}
\usepackage{ifxetex,ifluatex}
\usepackage{fixltx2e} % provides \textsubscript
\ifnum 0\ifxetex 1\fi\ifluatex 1\fi=0 % if pdftex
  \usepackage[T1]{fontenc}
  \usepackage[utf8]{inputenc}
\else % if luatex or xelatex
  \ifxetex
    \usepackage{mathspec}
  \else
    \usepackage{fontspec}
  \fi
  \defaultfontfeatures{Ligatures=TeX,Scale=MatchLowercase}
\fi
% use upquote if available, for straight quotes in verbatim environments
\IfFileExists{upquote.sty}{\usepackage{upquote}}{}
% use microtype if available
\IfFileExists{microtype.sty}{%
\usepackage{microtype}
\UseMicrotypeSet[protrusion]{basicmath} % disable protrusion for tt fonts
}{}
\usepackage[margin=1in]{geometry}
\usepackage{hyperref}
\hypersetup{unicode=true,
            pdftitle={LionfishDiet},
            pdfauthor={Alex Reulbach},
            pdfborder={0 0 0},
            breaklinks=true}
\urlstyle{same}  % don't use monospace font for urls
\usepackage{graphicx,grffile}
\makeatletter
\def\maxwidth{\ifdim\Gin@nat@width>\linewidth\linewidth\else\Gin@nat@width\fi}
\def\maxheight{\ifdim\Gin@nat@height>\textheight\textheight\else\Gin@nat@height\fi}
\makeatother
% Scale images if necessary, so that they will not overflow the page
% margins by default, and it is still possible to overwrite the defaults
% using explicit options in \includegraphics[width, height, ...]{}
\setkeys{Gin}{width=\maxwidth,height=\maxheight,keepaspectratio}
\IfFileExists{parskip.sty}{%
\usepackage{parskip}
}{% else
\setlength{\parindent}{0pt}
\setlength{\parskip}{6pt plus 2pt minus 1pt}
}
\setlength{\emergencystretch}{3em}  % prevent overfull lines
\providecommand{\tightlist}{%
  \setlength{\itemsep}{0pt}\setlength{\parskip}{0pt}}
\setcounter{secnumdepth}{0}
% Redefines (sub)paragraphs to behave more like sections
\ifx\paragraph\undefined\else
\let\oldparagraph\paragraph
\renewcommand{\paragraph}[1]{\oldparagraph{#1}\mbox{}}
\fi
\ifx\subparagraph\undefined\else
\let\oldsubparagraph\subparagraph
\renewcommand{\subparagraph}[1]{\oldsubparagraph{#1}\mbox{}}
\fi

%%% Use protect on footnotes to avoid problems with footnotes in titles
\let\rmarkdownfootnote\footnote%
\def\footnote{\protect\rmarkdownfootnote}

%%% Change title format to be more compact
\usepackage{titling}

% Create subtitle command for use in maketitle
\newcommand{\subtitle}[1]{
  \posttitle{
    \begin{center}\large#1\end{center}
    }
}

\setlength{\droptitle}{-2em}

  \title{LionfishDiet}
    \pretitle{\vspace{\droptitle}\centering\huge}
  \posttitle{\par}
    \author{Alex Reulbach}
    \preauthor{\centering\large\emph}
  \postauthor{\par}
      \predate{\centering\large\emph}
  \postdate{\par}
    \date{7/17/2018}


\begin{document}
\maketitle
\begin{abstract}
Lionfish (Pterois volitans and Pterois miles) are an invasive species in
the Western Atlantic that are causing harm to the region's coral reefs.
Previous studies have examined invasive lionfish feeding ecology in
large regions and suggested further studies identifying regional
variations and calling for more basic research regarding lionfish diet.
With limited resources, it is important to account for the spatial scale
at which variation in lionfish diet is reduced. In this study, 109
lionfish stomachs were collected from 10 sites in a localized region of
the Mexican Caribbean and were analyzed to evaluate the differences in
feeding ecology between sites. It was hypothesized that there would be
no noticeable differences between the feeding ecology of the lionfish in
the different sites across this region. Preliminary examination of the
data collected shows that no major differences between the feeding
ecology of the lionfish collected from the 10 different sites exist.
This finding is important for future effective management of invasive
lionfish populations across the Western Atlantic region. Since lionfish
feeding ecology does not change across different sites in a localized
region, studying lionfish at only one site in a certain region would be
effective for understanding lionfish feeding ecology across the entire
region, thus saving money which could be allocated for managing invasive
lionfish populations.
\end{abstract}

\section{Intro}\label{intro}

The invasion of lionfish (Pterois volitans and Pterois miles) into the
Western Atlantic represents a major threat to the coral reef ecosystems
in the region. Introduced in the early 1990's, the range of lionfish in
this region quickly spread up the East Coast, the Gulf of Mexico, and
throughout the Caribbean Sea (Betancur-R et al., 2011; Schofield, 2010).
The unique hunting strategy and appearance of lionfish account for their
success in establishing populations across the region (Grieve et al.,
2016). The high predation efficiency of the lionfish, due to the naivety
of native fish, has the potential to have large, negative ecological
effects on coral reefs fish and invertebrate species. Lionfish are
opportunistic generalist carnivores that consume a wide range of
invertebrate and vertebrate species. The majority of the prey species
consumed by invasive lionfish are not at risk or managed by the
fisheries (Peake et al., 2018). Since lionfish are opportunistic
generalist carnivores, they consume whatever prey species is most
abundant in a certain region. While the less abundant at risk species
are not being targeted by lionfish, this opportunistic hunting behavior
is still a cause for concern. In coral reefs with established lionfish
populations, the lionfish have caused significant reductions in the
recruitment of coral reef species (Albins and Hixon, 2008). For this
reason, understanding the composition of lionfish diet is crucial in
determining their effect on coral reef ecosystems. Invasive lionfish
feeding ecology and behavior has been examined before in numerous
location-based studies. These studies highlighted the many differences
in lionfish diet composition at different regions in the Western
Atlantic (Morris and Atkins, 2009; Eddy et al., 2016; Dahl and
Patterson, 2014). To discern if lionfish diet does vary based on
location, a large study examining the contents of 8125 lionfish stomachs
was conducted across 10 different locations in the Western Atlantic. The
study revealed that lionfish diet does vary considerably based on
location, and highlight the need for increased research to further
understand possible impacts of lionfish (Peake et al., 2018). While
lionfish diet varies considerably across large regions, it is unknown at
what spatial scale the variation in lionfish diet is reduced. Here, we
used stomach contents of 109 lionfish collected from 10 different sites
in a localized region of the Mexican Caribbean to examine the variation
in lionfish diet between sites and have a better understanding of
lionfish feeding ecology in this region. Based on previous
location-based studies that have examined lionfish feeding ecology in
other localized regions, we hypothesized that there would be very little
variation in lionfish diet and feeding ecology at the different sites.
The data collected from this study will be very important in
successfully managing invasive lionfish populations.

\section{Methods}\label{methods}

code that loads coastline code that creates map map

\section{Results}\label{results}

\section{Discussion}\label{discussion}

\section{References}\label{references}

\section{Acknowledgments}\label{acknowledgments}


\end{document}
